\chapter{Conclusion}
\label{ch:conclusion}
The results presented in the previous chapter show that the isolation overhead 
of process containerisation through resource namespacing is negligible. Disk latencies and throughput measurements 
in containers are on par with the same measurements on the host.
The networking infrastructure required to enable inter-container communication may lead to 
packet loss, higher retransmission counts and therefore increased latencies and decreased throughput.
Nevertheless, the average latency and throughput over the lifespan of a workload are almost
equivalent to running the application on the host. Furthermore, it is shown that 
hardware virtualisation mechanisms impact the resulting latencies and throughput far more 
severely \cite{https://doi.org/10.1002/cpe.5693} \cite{8457798}.
From a business perspective, the near-native performance of containers makes them highly 
favourable for consolidating multiple applications on a single server. The primary 
challenges with containers, however, lie not in performance, but rather security.

Resource namespaces encapsulate a variety of kernel resources, but not all of them.
Namespace-unaware kernel subsystems resort to authorising operations with the user identifiers 
in the root user namespace, i.e those that were mapped by the container runtime into the container.
Containers that share these identifiers implicitly share various in-kernel data structures 
that may lead to interference. For example, the kernel used to (up to v5.13.19) impose per-user resource consumption 
restrictions such as the total number of file descriptors that can be kept open by that user, the number of 
pending signals that can be queued among all processes of that user, and the number of processes 
that the user can spawn. If one container exhausts these limits, other containers poentially owned by a different tenant are directly affected, i.e 
interfered with. This particular problem has been fixed in the v5.14 release by binding resource limit counters to user namespaces
\cite{https://patchwork.kernel.org/project/linux-hardening/cover/cover.1619094428.git.legion@kernel.org/}.
Another very important example are filesystem implementations. Filesystems associate every file 
with a unique user identifier that is its owner. Root filesystems for containers residing on the 
host are typically owned by a single user and group (allocated by the container engine) and are inaccessible by other users 
to prevent tampering. Hence, to meaningfully mount such a filesystem in multiple rootless containers, 
each container's identifier map must point to that user on the host, i.e all containers must 
share the same user identifier. This problem has been tackled by traversing the entire directory hierarchy and changing the 
ownership of each node to the root user within the container after the mount - an error-prone operation 
that can take approximately 40 seconds for a large enough root filesystem \cite{https://github.com/containerd/containerd/pull/4734}.
The problem was addressed by allowing container runtimes to change file and directory ownerships on a per-mount basis - by remapping 
the 

\cite{https://lwn.net/Articles/896255/}.

Both examples show that kernel developers are actively working on supporting container software in user-space.