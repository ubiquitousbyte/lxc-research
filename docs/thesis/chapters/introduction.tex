\chapter{Introduction}
\section{Motivation}
Primitive support for multiprocessing in the form of basic context switching and dedicated input-output 
components was introduced in the late 1950s. Multiprocessing allowed for concurrent execution of 
multiple instructions at the cost of increased system complexity. Interleaved processes had a 
global unrestricted view of the system which inevitably led to unpredictable program behaviour. 
For example, programs had the ability to modify each other's memory and monopolise 
computer resources. Hence, to ensure correctness, every program had to carefully manage its interactions 
with hardware and all other processes in the system, which resulted in an unsustainable 
programming model.

The aforementioned issues were addressed by shifting the responsibility of resource management 
and process protection into a privileged control program that acted as an intermediary between 
hardware and user programs. This program was most commonly referred to as a kernel. 

\section{Objectives}
The primary objective of this thesis is to conduct an experiment consisting of a set of 
synthetic workloads that measure response times and efficiency within a sandboxed environment.
The experiment executes the same workloads natively, within the standard noninterference boundary 
provided by the kernel to user space programs. The differences in response times and efficiency 
within and outside a container are used as an approximation of the overhead introduced 
by isolating a process from its environment. The experiment is reproducible and can be executed 
on any Linux system with a x86\_64 or arm64 compatible processing unit.

The second objective of this thesis is to design and implement a container runtime 
that will be used by the experiment as the primary sandboxing mechanism. The runtime's 
implementation, including the system call that it relies on, will be thoroughly discussed and its security characteristics
will be qualitatively evaluated. 

\section{Content Structure}
Chapter \ref{ch:fundamentals} introduces the fundamental axioms, or trade-offs, in virtualisation technologies.
These will be referred to throughout the entire document. 
The same chapter also introduces the concept of resource namespaces - the primary abstraction 
provided by the kernel to sandbox processes.

Chapter \ref{ch:state-of-research} outlines the current state of research in operating-system 
virtualisation, highlights the relationships between noninterference, isolation and performance, and
discusses modern architectures that try to maximise all three.     

Chapter \ref{ch:concept} provides a detailed description of the requirements and the software architecture 
of the runtime, benchmark tool, and the workloads. The measurements to be sampled from the kernel are introduced.

Chapter \ref{ch:implementation} describes the implementation of the container runtime. Furthermore,
the security characteristics of the kernel's resource namespacing capabilities are discussed.

Chapter \ref{ch:Experiment} contains the performance evaluation experiment. First, the hardware 
platform under test is discussed. Second, a set of hypotheses are developed and proven and/or 
disproven with the help of the benchmark tool.  

Chapter \ref{ch:conclusion} contains conclusive remarks, a brief summary of the results of the thesis
and future work that can be done by other students in the field. 