\chapter{Introduction}
\section{Motivation}
Primitive support for multiprocessing in the form of basic context switching and dedicated input-output 
components was introduced in the late 1950s. Multiprocessing allowed for concurrent execution of 
multiple instructions at the cost of increased system complexity. Interleaved processes had a 
global unrestricted view of the system which inevitably led to unpredictable program behaviour. 
For example, programs had the ability to modify each other's memory and monopolise 
computer resources. Hence, to ensure correctness, every program had to carefully manage its interactions 
with hardware and all other processes in the system, which resulted in an unsustainable 
programming model.

The aforementioned issues were addressed by shifting the responsibility of resource management 
and process protection into a privileged control program that acted as an intermediary between 
hardware and user programs. This program was most commonly referred to as a kernel. 

\section{Objectives}
The primary objective of this thesis is to create a set of synthetic workloads and 
measure their performance characteristics when deployed inside a sandboxed environment.
The same workloads will also be executed natively within the standard noninterference boundary
provided by the kernel. The measurements will be compared and evaluated in an attempt to 
outline the relationship between noninterference, isolation and performance. 
A command-line tool will be developed to run the workloads and generate the measurements
so that the results are reproducible.

A custom container runtime will be designed, implemented and used by the command-line tool
as the primary sandboxing mechanism. The system call interface used for sandboxing 
the workloads will be discussed. The overhead introduced by the runtime to spawn 
a container will also be compared to the overhead associated with spawning a simple process.

\section{Content Structure}
Chapter \ref{ch:fundamentals} introduces the fundamental axioms, or trade-offs, in virtualisation technologies.
These will be referred to throughout the entire document. 
The same chapter also introduces the concept of resource namespaces - the primary abstraction 
provided by the kernel to sandbox processes. Furthermore, the extended Berkley Packet Filter (eBPF)
subsystem of the kernel is described, which we use to probe metrics from the kernel
that measure workload performance.

Chapter \ref{ch:state-of-research} outlines the current state of research in operating-system 
virtualisation, highlights the relationships between noninterference, isolation and performance, and
discusses modern architectures that try to maximise all three.     

Chapter \ref{ch:concept} provides a detailed description of the requirements and the software architecture 
of the runtime, benchmark tool, and the workloads. The measurements to be sampled from the kernel are introduced.

Chapter \ref{ch:implementation} describes the system calls used to create the sandbox environment.
Important implementation details are also mentioned.

Chapter \ref{ch:results} shows the results of the performance measurements and highlights the 
differences between running the workloads inside and outside a sandbox. The results are evaluated 
and the primary performance bottlenecks are discussed.

Chapter \ref{ch:conclusion} contains conclusive remarks, a brief summary of the results of the thesis
and future work that can be done by other students in the field. 