\chapter{Introduction}
\section{Motivation}
Primitive support for multiprocessing in the form of basic context switching and dedicated input-output 
components was introduced in the late 1950s. Multiprocessing allowed for concurrent execution of 
multiple instructions at the cost of increased system complexity. Interleaved processes had a 
global unrestricted view of the system which inevitably led to unpredictable program behaviour. 
For example, programs had the ability to modify each other's memory and monopolise 
computer resources. Hence, to ensure correctness, every program had to carefully manage its interactions 
with hardware and all other processes in the system, which resulted in an unsustainable 
programming model.

The aforementioned issues were addressed by shifting the responsibility of resource management 
and process protection into a privileged control program that acted as an intermediary between 
hardware and user programs.

TODO
\section{Objectives}
TODO
\section{Content Structure}
TODO