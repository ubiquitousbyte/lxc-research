\chapter*{Abstract}
Linux containers are the de-facto technology underlying 
modern application sandboxing and orchestration. This work focuses 
on analysing the primary isolation abstraction that makes 
containers possible - resource namespaces - and to qualitatively examine their security characteristics
by developing a proof-of-concept container runtime that supports rootless containers. 
Implementation details are provided and the system call interface is discussed.
In addition, the isolation overhead 
of the runtime with respect to networking and filesystem I/O is quantified and presented.
The results show that containers achieve native performance when interacting with 
disk devices, but may suffer from decreased network throughput and increased lantencies due to high variance 
and packet loss when interacting with each other from different network namespaces over a bridge device.
\\
\\
\textbf{Keywords:} Linux, Namespaces, Containers, System calls, Isolation, Performance
